%% The first command in your LaTeX source must be the \documentclass command.
%% Options: hf: enable header and footer.
\documentclass[
% twocolumn,
% hf,
]{ceurart}

%% One can fix some overfulls
%\sloppy


%% Minted listings support 
%% Need pygment <http://pygments.org/> <http://pypi.python.org/pypi/Pygments>
\usepackage{minted}
%% auto break lines
\setminted{breaklines=true}

%%
%% end of the preamble, start of the body of the document source.
\begin{document}

%%
%% Rights management information.
%% CC-BY is default license.
\copyrightyear{2025} % Actualizado para REST-MEX 2025
\copyrightclause{Copyright for this paper by its authors.
  Use permitted under Creative Commons License Attribution 4.0
  International (CC BY 4.0).}

%%
%% This command is for the conference information
\conference{}


%%
%% The "title" command
\title{Modelos Multimodales para Visual Question Answering en Imágenes Histopatológicas} 
% \subtitle{Clasificación de Polaridad, Tipo de Destino e Identificación de Pueblos Mágicos} % Subtítulo opcional

% \tnotemark[1] % Si necesitas una nota para el título
% \tnotetext[1]{Este trabajo fue parcialmente financiado por XYZ.}

%%
%% The "author" command and its associated commands are used to define
%% the authors and their affiliations.

% --- Equipo ---
\author[1]{Uziel Lujan Lopez}[
orcid=0009-0001-3160-5294, 
email=uziel.lujan@cimat.mx, 
]\fnmark[1]

\author[1]{Diego Paniagua-Molina}[
orcid=0009-0006-6564-2794, 
email=diego.paniagua@cimat.mx, 
]\fnmark[1]


% --- Afiliaciones ---
\address[1]{Mathematics Research Center (CIMAT—Centro de Investigación en Matemáticas), Graduate Program in Statistical Computing, Nuevo León, Mexico}



%% Footnotes
%\cortext[1]{Corresponding author.}
\fntext[1]{Estos autores contribuyeron de igual manera.}


%% Maximo 250 palabras, donde se describa el problema abordado, la base de datos empleada, los métodos utilizados, los resultados obtenidos y las conclusiones principales.
\begin{abstract}
  .
\end{abstract}

%%
%% Entre 3 y 5.
\begin{keywords}
  .
\end{keywords}

\maketitle

%--------------------------------------------------------------------------
%	CUERPO DEL ARTÍCULO
%--------------------------------------------------------------------------

% Incluir al menos una pregunta de investigación y formular los objetivos del proyecto bajo la metodologı́a SMART (especı́ficos, medibles, alcanzables, relevantes y con lı́mite de tiempo).
\section{Introducción}


\section{Gestión y Planeación del Proyecto}

% Roles y responsabilidades
\subsection{Matriz RACI}

% Diagrama de actividades
\subsection{Método PERT}

% WBS
\subsection{Estructura de Desglose del Trabajo}

% Identificación de la ruta crı́tica dentro del cronograma.
\subsection{Ruta Cronograma}

% Explicación de su uso como estrategia estructurada de resolución de problemas e innovación basada en principios inventivos.
\section{Metodología TRIZ}

% Revisión breve de soluciones previas similares.
\section{Trabajos Relacionados}


% Descripción detallada de la colección utilizada (origen, caracterı́sticas, tamaño, formato, preprocesamiento, etc.).
\section{Base de Datos}

% Descripción de la arquitectura, técnicas y herramientas empleadas.
\section{Propuesta de modelo desarrollado}

% Incluir métricas, visualizaciones, gráficas y tablas relevantes. Debes proponer un baseline muy sencillo para comparar los resultados.
\section{Resultados}

% Análisis crı́tico de los resultados, limitaciones del enfoque y posibles lı́neas de mejora.
\section{Discusión} 

% Sı́ntesis final de hallazgos y aportaciones.
\section{Conclusiones} 

% Bajo un formato académico uniforme (APA o IEEE).
\bibliography{refs.bib}

%%
%% If your work has an appendix, this is the place to put it.
% \appendix
% \section{Título del Apéndice A}
% ...

\end{document}
%%
%% End of file